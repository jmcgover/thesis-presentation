%%%%%%%%%% GRAPH %%%%%%%%%%
\begin{figure}
\centering
\pgfdeclarelayer{points} % declare foreground layer
\pgfdeclarelayer{edges} % declare foreground layer
\pgfdeclarelayer{grid} % declare foreground layer
\pgfsetlayers{grid,edges,points} % set layer order


\begin{subfigure}{0.45\linewidth}
\begin{tikzpicture}[scale=\knngraphscale, 
                    every node/.style={scale=\knngraphscale}]
    \begin{pgfonlayer}{grid}
    \draw[step=1, gray, very thin] (0-\gridextra,0-\gridextra) grid (\coordsize+\gridextra,\coordsize+\gridextra);
    \draw[thick, ->] (0,0) -- (\coordsize+\gridextra,0);
    \draw[thick, ->] (0,0) -- (0,\coordsize+\gridextra);
    \foreach \x in {1,...,10} % edit here for the numbers
    \draw[shift={(\x,0)},color=black] (0pt,0pt) -- (0pt,-3pt) node[below, fill=white] {$\x$};
    \foreach \y in {1,...,10} % edit here for the numbers
    \draw[shift={(0,\y)},color=black] (0pt,0pt) -- (-3pt,0pt) node[left, fill=white] {$\y$};
    \end{pgfonlayer}
    
    \begin{pgfonlayer}{points}
    \drawunknownpoint{5,5}{unknown};
    \drawbpoint{6,6}{one}; % 1.414 B
    \drawapoint{3,6}{two}; % 2.236 A
    \drawapoint{4,7}{three}; % 2.236 A
    \drawcpoint{3,3}{four}; % 3.000 C
    \drawbpoint{8,5}{five}; % 2.828 B
    \drawbpoint{6,8}{six}; % 3.162 B
    \drawcpoint{3,2}{seven}; % 3.606 C
    \drawcpoint{1,3}{eight}; % 4.472 C
    \drawcpoint{1,1}{nine}; % 5.657 C
    \drawapoint{1,10}{ten}; % 6.403 A
    \end{pgfonlayer}
    
    \begin{pgfonlayer}{edges}
        
        \path[-]
            (unknown)
            edge[]
            (one)
            
            (unknown)
            edge[]
            (two)
            
            (unknown)
            edge[]
            (three)
            ;
            
        
        \path[-]
            (unknown)
            edge[]
            (five)
            
            (unknown)
            edge[]
            (six)
            ;
        
        \path[-]
            (unknown)
            edge[]
            (seven)
            
            (unknown)
            edge[color=red]
            (eight)
            
            (unknown)
            edge[color=red]
            (nine)
            
            (unknown)
            edge[]
            (four)
            ;
    \end{pgfonlayer}
    
\end{tikzpicture}
\end{subfigure}
\hspace{23pt}
\begin{subfigure}{0.45\linewidth}
\small
\begin{tabular}{|c|c|c|c|}             \hline
  \rowcolor{white} \k{} & Class         & Position & Similarity    \\ \hline
  \rowcolor{white}      & \unknownchar{}& (5,5)     & 0.000       \\ \hline
  \hline
  \rowcolor{\bcolor}  1    & \bchar{}      & (6,6) & 1.414       \\ \hline
  \rowcolor{\acolor}  \ac2    & \ac\achar{}      & \ac(3,6) & \ac2.236       \\ \hline
  \rowcolor{\acolor}  \ac3    & \ac\achar{}      & \ac(4,7) & \ac2.236       \\ \hline
  \rowcolor{\ccolor}  4    & \cchar{}      & (3,3) & 2.828       \\ \hline
  \rowcolor{\bcolor}  5    & \bchar{}      & (8,5) & 3.000       \\ \hline
  \rowcolor{\bcolor}  6    & \bchar{}      & (6,8) & 3.162       \\ \hline
  \rowcolor{\ccolor}  7    & \cchar{}      & (3,2) & 3.606       \\ \hline
  \rowcolor{red}  8    & \cchar{}      & (1,3) & 4.472       \\ \hline
  \rowcolor{red}  9    & \cchar{}      & (1,1) & 5.657       \\ \hline
  %\rowcolor{\acolor} \ac10    & \ac\achar{}      & \ac(1,10) & \ac6.403       \\ \hline
\end{tabular}
\end{subfigure}
%\caption{Example graph of datapoints in a coordinate space. 
%All but one of the datapoints have a class associated with them, \achar{}, \bchar{}, or \cchar{}. 
%The class of one point, denoted by \unknownchar{}, requires determination. 
%\Compfuncs{} like \euclid{} or \manhattan{} may be the most appropriate way to compare these datapoints and \autoref{fig:knn} shows how \euclid{} and various \k{} can classify this point.}
\label{fig:knn-graph-alpha}
\end{figure}