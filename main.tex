% Copyright 2004 by Till Tantau <tantau@users.sourceforge.net>.
%
% In principle, this file can be redistributed and/or modified under
% the terms of the GNU Public License, version 2.
%
% However, this file is supposed to be a template to be modified
% for your own needs. For this reason, if you use this file as a
% template and not specifically distribute it as part of a another
% package/program, I grant the extra permission to freely copy and
% modify this file as you see fit and even to delete this copyright
% notice. 

\documentclass{beamer}
%%%%% THEME BEGIN %%%%%
% There are many different themes available for Beamer. A comprehensive
% list with examples is given here:
% http://deic.uab.es/~iblanes/beamer_gallery/index_by_theme.html
% You can uncomment the themes below if you would like to use a different
% one:
%\usetheme{AnnArbor}
%\usetheme{Antibes}
%\usetheme{Bergen}
%\usetheme{Berkeley}
%\usetheme{Berlin}
%\usetheme{Boadilla}
%\usetheme{boxes}
%\usetheme{CambridgeUS}
%\usetheme{Copenhagen}
%\usetheme{Darmstadt}
%\usetheme{default}
%\usetheme{Frankfurt}
%\usetheme{Goettingen}
%\usetheme{Hannover}
%\usetheme{Ilmenau}
%\usetheme{JuanLesPins}
%\usetheme{Luebeck}
%\usetheme{Madrid}
%\usetheme{Malmoe}
%\usetheme{Marburg}
%\usetheme{Montpellier}
%\usetheme{PaloAlto}
%\usetheme{Pittsburgh}
%\usetheme{Rochester}
%\usetheme{Singapore}
%\usetheme{Szeged}
\usetheme{Warsaw}

%%%%% CAL POLY COLOR DEFINITIONS
\usepackage{calpolycolors}

%%%%% CHANGING CRAP
\makeatletter % Makes @ a usable character (code 11?)

%%%%% OVERALL COLOR SCHEME
\setbeamercolor{structure}{fg=calpoly@cpgreen}

%%%%% HEADLINE
\setbeamertemplate{headline}{}

%%%%% TITLE
\setbeamercolor*{titlelike}{fg=white, bg=calpoly@cpgreen}
\setbeamercolor{frametitle}{fg=white,bg=calpoly@cpgreen}
\setbeamercolor{frametitle right}{fg=white,bg=calpoly@cpgreen}

%%%%% ROUNDED TITLE RECTANGLE WITH SHADOW
%\useinnertheme[shadow=true]{rounded}
\setbeamertemplate{title page}[default][colsep=-4bp,rounded=true,shadow=true]
\setbeamertemplate{sections/subsections in toc}[default]
\setbeamertemplate{itemize items}[default]
\setbeamertemplate{enumerate items}[default]

%%%%% SECTION 
%\setbeamercolor{section in toc}{fg=calpoly@cpgreen}
\setbeamercolor{section in head/foot}{bg=mcgovern@cpgold}

%%%%% ITEMIZE & ENUMERATE
\setbeamercolor{itemize item}{fg=mcgovern@cpgold}
\setbeamercolor{itemize subitem}{parent=itemize item}
\setbeamercolor{itemize subsubitem}{parent=itemize subitem}

%%%%% BLOCK TITLES
%\setbeamercolor{block title}{fg=calpoly@cpgreen}
%\setbeamercolor{block title example}{fg=calpoly@brightgold}

%%%%% NAVIGATION BUTTONS
\setbeamercolor{navigation symbols}{fg=calpoly@cpgreen, bg=mcgovern@cpgold}

\makeatother % Makes @ a it's normal character (code 12?)

%\usepackage{fontspec}
%\setmainfont{Avenir}
%%%%% THEME END %%%%%

%%%%% COLORS
%\usepackage{xcolor}
\definecolor{cpgreen}{HTML}{0A7951}
\definecolor{cpgold}{HTML}{FADA5E}

\usepackage{nth}

\usepackage{cplop}

\title[Investigating The \krap{} And Clustering For \bs{}]{Investigating The \kraplong{} And Clustering For \BSlongs{}}

% A subtitle is optional and this may be deleted
\subtitle{Exploring Two \MSTlong{} Methodologies}

\author{Jeffrey D. McGovern}
\institute{
  \textbf{\cplong{}}
  \\
  Computer Science and Software Engineering Department
}
\date{Master's Thesis Defense\\March \nth{21} 2016}

\subject{\MSTlong{}}
\pgfdeclareimage[height=0.5cm]{university-logo}{cp-logo-lrg}
\logo{\pgfuseimage{university-logo}}

%%%%% Table Contents %%%%%
\AtBeginSubsection[]
{
  \begin{frame}<beamer>{Outline}
    \tableofcontents[currentsection,currentsubsection]
  \end{frame}
}

%%%%% BEGIN %%%%%
\begin{document}

\begin{frame}
  \titlepage
\end{frame}

\begin{frame}{Outline}
  \tableofcontents
  % You might wish to add the option [pausesections]
\end{frame}

% Section and subsections will appear in the presentation overview
% and table of contents.
\section{First Main Section}

\subsection{First Subsection}

\begin{frame}{First Slide Title}{Optional Subtitle}
  \begin{itemize}
  \item {
    My first point.
  }
  \item {
    My second point.
  }
  \end{itemize}
\end{frame}

\subsection{Second Subsection}

% You can reveal the parts of a slide one at a time
% with the \pause command:
\begin{frame}{Second Slide Title}
  \begin{itemize}
  \item {
    First item.
    \pause % The slide will pause after showing the first item
  }
  \item {   
    Second item.
  }
  % You can also specify when the content should appear
  % by using <n->:
  \item<3-> {
    Third item.
  }
  \item<4-> {
    Fourth item.
  }
  % or you can use the \uncover command to reveal general
  % content (not just \items):
  \item<5-> {
    Fifth item. \uncover<6->{Extra text in the fifth item.}
  }
  \end{itemize}
\end{frame}

\section{Second Main Section}

\subsection{Another Subsection}

\begin{frame}{Blocks}
\begin{block}{Block Title}
You can also highlight sections of your presentation in a block, with it's own title
\end{block}
\begin{theorem}
There are separate environments for theorems, examples, definitions and proofs.
\end{theorem}
\begin{example}
Here is an example of an example block.
\end{example}
\end{frame}

% Placing a * after \section means it will not show in the
% outline or table of contents.
\section*{Summary}

\begin{frame}{Summary}
  \begin{itemize}
  \item
    The \alert{first main message} of your talk in one or two lines.
  \item
    The \alert{second main message} of your talk in one or two lines.
  \item
    Perhaps a \alert{third message}, but not more than that.
  \end{itemize}
  
  \begin{itemize}
  \item
    Outlook
    \begin{itemize}
    \item
      Something you haven't solved.
    \item
      Something else you haven't solved.
    \end{itemize}
  \end{itemize}
\end{frame}



% All of the following is optional and typically not needed. 
\appendix
\section<presentation>*{\appendixname}
\subsection<presentation>*{For Further Reading}

\begin{frame}[allowframebreaks]
  \frametitle<presentation>{For Further Reading}
    
  \begin{thebibliography}{10}
    
  \beamertemplatebookbibitems
  % Start with overview books.

  \bibitem{Author1990}
    A.~Author.
    \newblock {\em Handbook of Everything}.
    \newblock Some Press, 1990.
 
    
  \beamertemplatearticlebibitems
  % Followed by interesting articles. Keep the list short. 

  \bibitem{Someone2000}
    S.~Someone.
    \newblock On this and that.
    \newblock {\em Journal of This and That}, 2(1):50--100,
    2000.
  \end{thebibliography}
\end{frame}

\end{document}


