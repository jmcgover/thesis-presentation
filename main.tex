% Copyright 2004 by Till Tantau <tantau@users.sourceforge.net>.
%
% In principle, this file can be redistributed and/or modified under
% the terms of the GNU Public License, version 2.
%
% However, this file is supposed to be a template to be modified
% for your own needs. For this reason, if you use this file as a
% template and not specifically distribute it as part of a another
% package/program, I grant the extra permission to freely copy and
% modify this file as you see fit and even to delete this copyright
% notice. 

%%%%% DOCUMENT CLASS %%%%%
% Alternatively Comment/Uncomment these two lines 
% to print out the slides without pause
\documentclass{beamer}
%\documentclass[handout]{beamer}

%%%%%%%%%%%%%%%%%%%%%%%
%%%%% THEME BEGIN %%%%%
% There are many different themes available for Beamer. A comprehensive
% list with examples is given here:
% http://deic.uab.es/~iblanes/beamer_gallery/index_by_theme.html
% You can uncomment the themes below if you would like to use a different
% one:
%\usetheme{AnnArbor}
%\usetheme{Antibes}
%\usetheme{Bergen}
%\usetheme{Berkeley}
%\usetheme{Berlin}
%\usetheme{Boadilla}
%\usetheme{boxes}
%\usetheme{CambridgeUS}
%\usetheme{Copenhagen}
%\usetheme{Darmstadt}
%\usetheme{default}
%\usetheme{Frankfurt}
%\usetheme{Goettingen}
%\usetheme{Hannover}
%\usetheme{Ilmenau}
%\usetheme{JuanLesPins}
%\usetheme{Luebeck}
%\usetheme{Madrid}
%\usetheme{Malmoe}
%\usetheme{Marburg}
%\usetheme{Montpellier}
%\usetheme{PaloAlto}
%\usetheme{Pittsburgh}
%\usetheme{Rochester}
%\usetheme{Singapore}
%\usetheme{Szeged}
%\usetheme{Warsaw}

%%%%% CAL POLY COLOR DEFINITIONS
%\usepackage{calpolycolors}
\usetheme{CalPoly}

%%%%% CHANGING CRAP
%\usepackage{fontspec}
%\setmainfont{Avenir}
%%%%% THEME END %%%%%
%%%%%%%%%%%%%%%%%%%%%

%%%%% PACKAGES %%%%%
\usepackage{nth}
\usepackage{hyperref}

%%%%% COLORS
%\usepackage{xcolor}
\definecolor{cpgreen}{HTML}{0A7951}
\definecolor{cpgold}{HTML}{FADA5E}

%%%%% CUSTOM MACROS %%%%%
\usepackage{cplop}

%%%%% TITLE/SUBTITLE %%%%%%
\title[Investigating The \krap{} And Clustering For \bs{}]{Investigating The \kraplong{} And Clustering For \BSlongs{}}
\subtitle{Exploring Two \MSTlong{} Methodologies}

%%%%% AUTHOR/INSTITUTE %%%%%%
\author{Jeffrey D. McGovern}
\institute{
  \textbf{\cplong{}}
  \\
  Computer Science and Software Engineering Department
}
%%%%% DATE/CONFERENCE %%%%%%
\date[March \nth{21} 2016]{Master's Thesis Defense\\March \nth{21} 2016}

%%%%% SUBJECT KEYWORDS %%%%%%
% Doesn't actually show up on the slides anywhere
\subject{\MSTlong{}}

%%%%% RECURRING LOGO %%%%%%
\pgfdeclareimage[height=0.5cm]{university-logo}{cp-logo-lrg}
\logo{\pgfuseimage{university-logo}}

%%%%% SHOW TABLE CONTENTS %%%%%
% This block creates a slide right when a new subsection starts.
% Comment or delete this block to remove this feature.
\AtBeginSubsection[]
{
\begin{frame}<beamer>{Outline}
    \tableofcontents[currentsection,currentsubsection]
\end{frame}
}

%%%%%%%%%%%%%%%%%%%%%%%%%%
%%%%% BEGIN DOCUMENT %%%%%
\begin{document}

%%%%% TITLE FRAME %%%%%
\begin{frame}
    \titlepage
\end{frame}

%%%%% TABLE OF CONTENTS / SECTION OUTLINE %%%%%
\begin{frame}{Outline}
    \tableofcontents
    % You might wish to add the option [pausesections]
\end{frame}

%%%%% SECTION: TOC VISIBLE %%%%%
% Section and subsections will appear in the presentation overview
% and table of contents.
\section{First Main Section}
\subsection{First Subsection}

%%%%% SLIDE WITH OPTIONAL SUBTITLE %%%%%
\begin{frame}{First Slide Title}{Optional Subtitle}
    \begin{itemize}
        \item {
            My first point.
        }
        \item {
            My second point.
        }
    \end{itemize}
\end{frame}

\subsection{Second Subsection}

%%%%% REVEALING PARTS OF SLIDES %%%%%
% You can reveal the parts of a slide one at a time
% with the \pause command:
\begin{frame}{Second Slide Title}
    \begin{itemize}
        \item {
            First item.
            \pause % The slide will pause after showing the first item
        }
        \item {   
            Second item.
        }
    % You can also specify when the content should appear
    % by using <n->:
    \item<3-> {
        Third item.
    }
    \item<4-> {
        Fourth item.
    }
    % or you can use the \uncover command to reveal general
    % content (not just \items):
    \item<5-> {
        Fifth item. \uncover<6->{Extra text in the fifth item.}
    }
    \end{itemize}
\end{frame}

\section{Second Main Section}

\subsection{Another Subsection}

%%%%% BLOCKS %%%%%
\begin{frame}{Blocks}
    % BLOCK
    \begin{block}{Block Title}
        You can also highlight sections of your presentation in a block, with it's own title
    \end{block}
    % THEOREM BLOCK
    \begin{theorem}
        There are separate environments for theorems, examples, definitions and proofs.
    \end{theorem}
    % EXAMPLE BLOCK
    \begin{example}
        Here is an example of an example block.
    \end{example}
    % ALERT BLOCK
    \begin{alertblock}{Alert Block}
        Here is an example of an alertblock block.
    \end{alertblock}
\end{frame}

%%%%% SECTION: TOC INVISIBLE %%%%%
% Placing a * after \section means it will not show in the
% outline or table of contents.
\section*{Summary}

%%%%% ITEMIZE & ENUMERATE %%%%%
\begin{frame}{Summary}
    \begin{itemize}
        \item
        The \alert{first main message} of your talk in one or two lines.
        \item
        The \alert{second main message} of your talk in one or two lines.
        \item
        Perhaps a \alert{third message}, but not more than that.
    \end{itemize}
    
    \begin{itemize}
        \item
        Outlook
        \begin{enumerate}
            \item
            Something you haven't solved \cite{DBLP:conf/bibm/McGovernDKBVG15, DBLP:conf/bcb/McGovernJDBKV16}.
            \item
            Something else you haven't solved.
            \item 
        \end{enumerate}
    \end{itemize}
\end{frame}



%%%%% BIBLIOGRAPHY %%%%%
\appendix
\section<presentation>*{\appendixname}
\subsection<presentation>*{For Further Reading}

% Good for stuff that doesn't need to go into a .bib file
\begin{frame}[allowframebreaks]
  \frametitle<presentation>{For Further Reading}
    \begin{thebibliography}{10}
        
        \beamertemplatebookbibitems
        % Start with overview books.
        
        \bibitem{Author1990}
        A.~Author.
        \newblock {\em Handbook of Everything}.
        \newblock Some Press, 1990.
        
        
        \beamertemplatearticlebibitems
        % Followed by interesting articles. Keep the list short. 
        
        \bibitem{Someone2000}
        S.~Someone.
        \newblock On this and that.
        \newblock {\em Journal of This and That}, 2(1):50--100,
        2000.
        
        \setbeamertemplate{bibliography item}[online]
        \bibitem{A} ItemA
        \setbeamertemplate{bibliography item}[book]
        \bibitem{B} ItemB
        \setbeamertemplate{bibliography item}[article]
        \bibitem{C} ItemC
        \setbeamertemplate{bibliography item}[triangle]
        \bibitem{D} ItemD
        \setbeamertemplate{bibliography item}[text]
        \bibitem{E} ItemE
    \end{thebibliography}
\end{frame}

%%%%% CITATIONS %%%%%
\begin{frame}[t, allowframebreaks]
    \tiny
    %\nocite{*}
    \frametitle<presentation>{References}
    \setbeamertemplate{bibliography item}[text]
    \bibliographystyle{plain}
    \bibliography{bibliography}
\end{frame}
\end{document}
%%%%% END DOCUMENT %%%%%
%%%%%%%%%%%%%%%%%%%%%%%%%%